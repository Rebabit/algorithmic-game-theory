%! TEX program = pdflatex

\documentclass[oneside,solution]{seu-ml-assign}
\usepackage{CJK}
\usepackage{graphicx}
\usepackage{subfigure}
\usepackage{float}
\usepackage{setspace}
\usepackage{indentfirst}
\setlength{\parindent}{2em}
\renewcommand{\baselinestretch}{1.25}
\title{Homework}
\author{}
\studentID{}
\instructor{Zhengyang Liu}
\date{\today}
\duedate{July 17, 2022}
\assignno{2}
\semester{BIT --- 2022 Spring}
\mainproblem{Mechanism Design}

\begin{document}

\maketitle
%\startsolution[print]
\problem{Favourite Movie}
\emph{Interstellar} directed by Christopher Nolan. It's a grand and
thoughtful journey into time and space and gravity, into the beautiful
and dazzling universe. But it goes beyond that, as "love is the one
thing that transcends time and space." It's a movie that awes me and
inspires me into the known.

Of course, \emph{A Beautiful Mind} is a movie worth watching, too.
\problem{Favourite Theorem in Class}
My favorite one is Myerson's Lemma. In the first place, the theorem
itself is powerful and concise: it turns the elusive "implementable"
goal into an explicit one with uniqueness and offers a simple formula to
achieve it. Furthermore, the lemma is useful
in many cases to attain DSIC mechanism, for example, revenue-maximization auction design. It  lays foundation for subsequent
theories and thus is of significant
importance. Most importantly, it sparks my interest and inspires to me
to learn more about mechanism design from that class on.
\problem{XOS Function}
(Discussed with Jingyi Gang, Sitian Shen)


To prove this, we will first construct \(2^m\) additive valuations for
the mapping of \(a_T(S), T\in [m]\) such that
\(v(S)=\max_{T\subseteq[m]}a_{T}(S)\). Then, using given information, we
will prove that any monotone and normalized submodular function \(v(S)\)
can be written as an XOS function using this way of construction.
\subsection{Construction of Additive
Valuations}
\\ \hspace*{\fill} \\
$\text{        }\quad\text{  }$
In this section, we construct \(2^m\) additive valuations such that
\(v(S)=a_{S}(S)\).

For \(S=\emptyset\), define \(a_{\emptyset}(\emptyset)\equiv0\). For \(i\in[m]\),
define \(a_{\emptyset}(i)\equiv0\)

For \(S\notin \emptyset,i\in[m]\), taking random selection of an element
\(j\) from the set \(S\), we define \(a_{S}(i)\) as follows:

\begin{equation}
a_S(i)=\left\{
\begin{aligned}
&-\infty,  & i\notin S \\
&a_{S-\{j\}}(i)  ,& i \neq j \\
&v(S)-v(S-\{j\}),& i=j
\end{aligned}
\right.
\end{equation}

As \(a_{S}\) is an additive function, using information that \(v\) is
normalized(i.e., \(v(\emptyset)=0\)), we can infer that
\begin{equation}
    \begin{aligned}
a_{S}(S)&=\sum_{i\in S}a_{S}(i)=a_{S}(j)+\sum_{i\in S-\{j\}}a_{S}(i)\\
&=v(S)-v(S-\{j\})+a_{S-\{j\}}(S-\{j\})=\cdots \\
&=v(S)-v(S-\{j\})+v(S-\{j\})\cdots-v(\{k\})+v(\{k\})-v(\emptyset)+a_{\emptyset}(\emptyset)\\
&=v(S)
\end{aligned}
\end{equation}


For \(T\in[m]\), we can obtain that

\begin{equation}
    a_T(S)=\left\{
\begin{aligned}
&-\infty,  &S\not\subseteq T \\
&a_{T-\{i\}}(S), &S\subseteq T-\{i\}  \\
&a_{T-\{i\}}(S)+v(T)-v(T-\{i\})  ,&\text{other} \\
\end{aligned}
\right.
\label{eq:4-3}
\end{equation}
\subsection{Proof of XOS}
\\ \hspace*{\fill} \\
$\text{        }\quad\text{  }$
In this section, using the construction above, we aim to prove that any
monotone and normalized submodular function \(v(S)\) can be written as
an XOS function. That is, \(v(S)=a_{S}(S)=max_{T\in[m]}(S)\).
\paragraph{case 1} \(S\not\subseteq T\)

From Eq.~\eqref{eq:4-3}, it is obvious
that \(a_{T}(S)=-\infty\). As \(v\) is monotone and normalized, we can
infer that \(v(S)\geq v(\emptyset)=0\). Thus,
\(a_{T}(S)=-\infty<a_{S}(S)=v(S)\).

\paragraph{case 2}
\(S \subseteq S+\{i\}\subseteq \cdots\subseteq T-\{j\} \subseteq T\)
when \(a_{T}(S)\) is inherited from \(a_{S}(S)\)

From Eq.~\eqref{eq:4-3}, we can know
that \(a_{T}(S)=a_{T-\{j\}}(S)=\cdots=a_{S+\{i\}}(S)=a_{S}(S)\).

\paragraph{case 3} \(S \subseteq T\), but not in case 2.

Suppose \(T=S+\{i\}\) and \(T\) is not inherited from \(S\), we can show
that \(a_T(S)\leq v(S)\).

Prove by mathematical induction.

For \(S=\{j\},|S|=1\) and \(T\) is inherited from \(\{i\}\)
\begin{equation}
    \begin{aligned}
    &a_T(i)=a_{\{i\}}(i)=v(\{i\})-v(\emptyset)=v(\{i\})\\
&a_T(j)=v(T)-v(\{i\})\\
&a_T(T)=v(T)
    \end{aligned}
\end{equation}

Recall that \(v\) is submodular, that is,
\begin{equation}
    v(s\cup t)+v(s\cap t)\leq v(s)+v(t)
    \label{equ:4-5}
\end{equation}

Thus, \(v(T)=v(\{i\}+\{j\})+v(\emptyset)<v(\{i\})+v(\{j\})\),
\begin{equation}
    a_T(S)=a_T(j)=v(T)-v(\{i\})<v({j})=v(S)
\end{equation}

For \(|S|\neq1\) and \(T\) is inherited from \(S-\{j\}\), assume that
\(a_{T-\{j\}}(S-\{j\})\leq v(S-\{j\})\), we can infer that
\begin{equation}
    \begin{aligned}
    a_{T}(j)&=v(T)-v(T-\{j\})\\
a_T(S)=a_T(T-\{i\})&=a_T(S-\{j\})+a_T(\{j\})\\
&\leq v(S-\{j\})+v(S+\{i\})-v(S+\{i\}-\{j\})
    \end{aligned}
\end{equation}

Use the submodular identity again, according to Eq.~\eqref{equ:4-5}. Set \(s=S\),
\(t=S+\{i\}-\{j\}\) and substitute the formula
\begin{equation}
    \begin{aligned}
    v(S+\{i\})+v(S-\{j\})\leq v(S)+v(S+\{i\}-\{j\})\\
a_T(S) \leq v(S+\{i\})+v(S-\{j\})-v(S+\{i\}-\{j\}) \leq
v(S)
    \end{aligned}
\end{equation}

Thus, for \(T=S+\{i\}\), we can show that \(a_T(S)\leq v(S)\).

                  For all the situations in case \(3\), we can prove that
\(a_T(S)\leq v(S)\) always exists using mathematical induction similar
to the previous one.

Hence, combining case 1,2 and 3, we can prove that any monotone and
normalized submodular function \(v(S)\) can be written as an XOS
function.

\problem{Virtual Valuation and Regularity Condition}
\subsection{}
For a uniform distribution, the distribution function is as follows:

\begin{equation}
    F(x)=x,\text{where }x\in[0,1]
\end{equation}

As \(F^{-1}(x)=x\), \(V(q)=F^{-1}(1-q)=1-q\), which denotes the posted
price resulting in a probability \(q\) of a sale.

Hence, the revenue curve \(R(q)=q\cdot V(q)=q(1-q)\).

\subsection{}\label{sec:4-2}
The revenue curve at \(q\)
\begin{equation}
    R(q)=q \cdot V(q)=q\cdot F^{-1}(1-q),q\in[0,1]
    \label{eq:3-2}
\end{equation}

Take the inverse function of \(F^{-1}\), we can obtain

\begin{equation}
    F(\dfrac{R(q)}{q})=1-q
    \label{eq:3-3}
\end{equation}

Then, take the derivative of both sides

\begin{equation}
    f(\frac{R(q)}{q})\cdot \frac{R^{'}(q)\cdot q-R(q)\cdot1}{q^2}=-1
    \label{eq:3-4}
\end{equation}

Based on equation Eq.~\eqref{eq:3-2}, we simplify Eq.~\eqref{eq:3-4} 
\begin{equation}
\begin{aligned}
f(V(q))\cdot\frac{R^{'}(q)\cdot q-q\cdot V(q)}{q^2}&=-1\\
f(V(q))\cdot\frac{R^{'}(q)-V(q)}{q}&=-1
\end{aligned}
\end{equation}

Thus, we get the slope of the revenue curve at \(q\)
\begin{equation}
    R^{'}(q)=V(q)-\frac{q}{f(V(q))}
\end{equation}

According to equation Eq.~\eqref{eq:3-3},
\(q=1-F(\frac{R(q)}{q})=1-F(V(q))\). Hence, \(R^{'}(q)=\varphi(V(q))\),
since the virtual valuation \(\varphi(v)=v-\frac{1-F(v)}{f(v)}\) and
\(V(q)\) can get all possible values of \(v\) for \(q\in[0,1]\).


\subsection{}\label{sec:4-3}
We have proved in Sec.~\eqref{sec:4-2} that
\begin{equation}
    R^{'}(q)=\varphi(V(q))
\end{equation}

Take the derivative of both sides
\begin{equation}
    R^{''}(q)=[\varphi(V(q))]^{'}=\varphi^{'}(V(q))\cdot V^{'}(q)
\end{equation}

Then, take the derivative of equation Eq.~\eqref{eq:3-2}, we can get the
derivative of \(V(q)\)
\begin{equation}
\begin{aligned}
f(V(q))\cdot V^{'}(q)=-1\\
V^{'}(q)=-\frac{1}{f(V(q))}
\end{aligned}
\end{equation}

Since the distribution density is strictly positive, \(V^{'}(q)<0\).
Thus,
\begin{equation}
    R^{''}(q)<0\;\text{ iff }\varphi^{'}(V(q))>0,q\in[0,1]
    \label{eq:3-10}
\end{equation}

Furthermore, for \(q\in[0,1]\), \(V(q)\) can get all possible values of
\(v\). Hence, equation \ref{eq:3-10} can be turned into
\begin{equation}
    R^{''}(q)<0\;\text{ iff }\varphi^{'}(v)>0,v\in[0,v_{max}]
\end{equation}

Thus, the revenue curve is concave if and only if \(\varphi(v)\) is a
monotone non-decreasing function for \(v \in [0,v_{max}]\), which is
equivalent to the definition of a regular distribution.
\subsection{}
As proven in section Sec.~\eqref{sec:4-3}, if a valuation distribution is regular, its
revenue curve is concave. That is to say, for \(q\in[0,1]\),
\(R^{''}(q)<0\).

Let \(q^{'}\) be the the maximum point of \(R\), namely
\(R(q^{'})= max_{q\in[0,1]}R(q)\). According to Jensen's inequality, the
secant line of a convex function lies above the graph of the function,
we can know that
\begin{equation}
    \begin{aligned}
    R\left(\frac{0+q^{'}}{2}\right)\geq \frac{R(0)+R(q^{'})}{2}\\
R\left(\frac{q^{'}+1}{2}\right)\geq \frac{R(q^{'})+R(1)}{2}
    \end{aligned}
    \label{eq:3-13}
\end{equation}

Since \(R(0)=R(1)=0\), Eq.~\eqref{eq:3-13} can be simplified as
follows:
\begin{equation}
    \begin{aligned}
    R\left(\frac{q^{'}}{2}\right)\geq\frac{1}{2}R(q^{'})\\
    R\left(\frac{q^{'}+1}{2}\right)\geq \frac{1}{2}R(q^{'})
    \end{aligned}
\end{equation}
As \(0\leq q^{'}\leq1\),
\(\dfrac{q^{'}}{2}\leq \dfrac{1}{2}\leq \dfrac{q^{'}+1}{2}\). According
to the properties of convex functions

\begin{equation}
    \begin{aligned}
    R\left(\frac{1}{2}\right)\geq R\left(\frac{q^{'}}{2}\right)\geq\frac{1}{2}R(q^{'})\\
R\left(\frac{1}{2}\right)\geq R\left(\frac{q^{'}+1}{2}\right)\geq \frac{1}{2}R(q^{'})\\
\Rightarrow R\left(\frac{1}{2}\right)\geq \frac{1}{2}max_{q\in[0,1]}R(q)
    \end{aligned}
\end{equation}

Thus, we can prove that the median price \(V\left(\dfrac{1}{2}\right)\)
is a \(\dfrac{1}{2}\)-approximation of the optimal posted price for a
regular distribution.

\end{document}
